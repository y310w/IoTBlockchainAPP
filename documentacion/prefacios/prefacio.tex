\myemptypage
\begin{titlepage}

\setlength{\centeroffset}{-0.5\oddsidemargin}
\addtolength{\centeroffset}{0.5\evensidemargin}

\noindent\hspace*{\centeroffset}\begin{minipage}{\textwidth}

\centering

%si el proyecto tiene logo poner aquí
\includegraphics[width=0.9\textwidth]{imagenes/logo_ugr}\\[1.4cm]

% Title

{\Huge\bfseries Aplicación Distribuida Basada en Blockchain para dar soporte a IoT\\
}
\noindent\rule[-1ex]{\textwidth}{3pt}\\[3.5ex]
{\large\bfseries Desarrollo de un registro distribuido para IoT usando la tecnología Blockchain.\\[4cm]}
\end{minipage}

\vspace{2.5cm}
\noindent\hspace*{\centeroffset}\begin{minipage}{\textwidth}
\centering

\textbf{Autor}\\ {Fernando Rafael Talavera Mendoza}\\[2.5ex]
\textbf{Director}\\
{Pedro Ángel Castillo Valdivieso}\\[2cm]

\includegraphics[width=0.2\textwidth]{imagenes/atc}\\[0.1cm]
\textsc{Departamento de Arquitectura y Tecnología de Computadores}\\
\textsc{---}\\
Granada, Noviembre de 2020
\end{minipage}
%\addtolength{\textwidth}{\centeroffset}
\end{titlepage}




\myemptypage
\thispagestyle{empty}

\begin{center}
{\large\bfseries Aplicación Distribuida Basada en Blockchain para dar soporte a IoT}\\
\end{center}
\begin{center}
Fernando Rafael Talavera Mendoza\\
\end{center}

%\vspace{0.7cm}
\noindent{\textbf{Palabras clave}: Blockchain, Raspberry Pi, Hyperledger Fabric, Docker, GraphQL, Gatsby.}

\vspace{0.7cm}
\noindent{\textbf{Resumen}}

\vspace{5mm}

Desde el comienzo de Bitcoin, la tecnología Blockchain surgió como la siguiente tecnología revolucionaria. 
Aunque Blockchain comenzó como una tecnología central de Bitcoin, sus casos de uso se están expandiendo a 
muchas otras áreas, incluyendo finanzas, Internet de las cosas, seguridad y demás. Cabe destacar, que la 
tecnología del Internet de las cosas ha tenido gran influencia y aceptación, y actualmente, muchos sectores 
públicos y privados se están sumergiendo en la tecnología. Estos dispositivos necesitan comunicarse y 
sincronizarse entre sí. Pero en situaciones en las que más de miles o decenas de miles de dispositivos que se 
conectan, esperamos que el uso del modelo actual de cliente-servidor pueda tener limitaciones y problemas 
durante la sincronización. 

\vspace{5mm}

\noindent Por tanto, se propone usar la tecnología Blockchain para controlar y configurar los dispositivos 
de IoT. Realizar un estudio del papel que puede tomar en dicho escenario, a nivel de soluciones, benefios 
y desafíos. Y escoger las herramientas para realizar una prueba de concepto con Raspberry Pi e implementar
una aplicación distribuida para facilitar el registro y la administración de los dispositivos a los 
usuarios en una red domótica.

\newpage
\thispagestyle{empty}

\begin{center}
{\large\bfseries Distributed Application Based on Blockchain to support IoT: Development of a distributed registry for 
IoT using Blockchain technology.}\\
\end{center}
\begin{center}
Fernando Rafael Talavera Mendoza\\
\end{center}

%\vspace{0.7cm}
\noindent{\textbf{Keywords}: Blockchain, Raspberry Pi, Hyperledger Fabric, Docker, GraphQL, Gatsby.}

\vspace{0.7cm}
\noindent{\textbf{Abstract}}

\vspace{5mm}

Since the start of Bitcoin in 2008, blockchain technology emerged as the next revolutionary technology. 
Though blockchain started off as a core technology of Bitcoin, its use cases are expanding to many other 
areas including finances, Internet of Things (IoT), security and such. It is noteworthy, that IoT technology 
has had great influence and acceptance, and currently, many private and public sectors are diving into the 
technology. These IoT devices need to communicate and synchronize with each other. But in situations where 
more than thousands or tens of thousands of IoT devices are connected, we expect that using current model of 
server-client may have some limitations and issues while in synchronization.

\vspace{5mm}

\noindent Therefore, it is proposed to use the Blockchain technology to control and configure the IoT devices. 
Make a study of the role it can take in that scenario, at the level of solutions, benefits and challenges. 
And choose the tools to perform a proof of concept with Raspberry Pi and implement a distributed application 
to make it easier for users to register and manage their devices in a home automation network.

\clearpage
\thispagestyle{empty}

\noindent\rule[-1ex]{\textwidth}{2pt}\\[4.5ex]

Yo, \textbf{Nombre Fernando Rafael Talavera Mendoza}, alumno de la titulación \textbf{Grado en Ingeniería Informática} 
de la \textbf{Escuela Técnica Superior de Ingenierías Informática y de Telecomunicación de la Universidad de Granada}, 
con DNI 77147088M, autorizo la ubicación de la siguiente copia de mi Trabajo Fin de Grado en la biblioteca del centro 
para que pueda ser consultada por las personas que lo deseen.

\vspace{6cm}

\noindent Fdo: Fernando Rafael Talavera Mendoza

\vspace{2cm}

\begin{flushright}
Granada a 18 de Noviembre de 2020.
\end{flushright}
\clearpage


\newpage
\thispagestyle{empty}

\noindent\rule[-1ex]{\textwidth}{2pt}\\[4.5ex]

D. \textbf{Pedro Ángel Castillo Valdivieso}, Profesor del Área Arquitectura y Tecnología de Computadores de la Universidad de Granada.

\vspace{0.5cm}

\textbf{Informan:}

\vspace{0.5cm}

Que el presente trabajo, titulado \textit{\textbf{Aplicación Distribuida Basada en Blockchain para dar soporte a IoT, Desarrollo de un registro distribuido 
para IoT usando la tecnología Blockchain}}, ha sido realizado bajo su supervisión por \textbf{Fernando Rafael Talavera Mendoza}, y autorizamos la 
defensa de dicho trabajo ante el tribunal que corresponda.

\vspace{0.5cm}

Y para que conste, expiden y firman el presente informe en Granada a 18 de Noviembre de 2020.

\vspace{1cm}

\textbf{Los directores:}

\vspace{5cm}

\noindent \textbf{Pedro Ángel Castillo Valdivieso}

\newpage
\thispagestyle{empty}

\section*{Agradecimientos}
\thispagestyle{empty}
\vspace{1cm}

A mi familia, por ser mi brújula que me guía. Mi inspiración para llegar a grandes alturas y mi consuelo ante los fallos.

\vspace{5mm}

A mis amigos, por estar ahí en todo momento y transmitirme buenos momentos y recuerdos.

\vspace{5mm}

Y gracias a mi tutor, Pedro, por animarme en este proyecto y confiar en mí.

\myemptypage
